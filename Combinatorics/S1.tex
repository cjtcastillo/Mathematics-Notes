\begin{theorem}[Addition Principle]
	Let $A_1, A_2, \dots, A_n$ be pairwise disjoint finite sets, so that
	\[
	A_i \cap A_j = \varnothing \quad \text{for } i \neq j.
	\]
	Then
	\[
	\left| \bigcup_{i=1}^n A_i \right|
	= \sum_{i=1}^n |A_i|.
	\]
\end{theorem}
\begin{remark}
	If the sets $A_1, \dots, A_n$ are not pairwise disjoint, the formula
	$\left|\bigcup_i A_i\right| = \sum_i |A_i|$ can fail because overlapping
	elements would be counted twice. Observe figure 1.2 for specific example. However, modification to the principle via compensation of the over-count corrects such issues:
	\begin{equation*}
		|A|+|B| - |A\cap B|
	\end{equation*}
\end{remark}
\begin{marginfigure}[-50mm]
	\begin{center}
		\includegraphics[scale=.5]{2count}
		\caption{Two sets $A$ and $B$ with overlap. In the specific figure, the principle would produce a 6 count when there are only really 5}.
	\end{center}
\end{marginfigure}
\vspace{-2mm}
\begin{proof}
	Since the sets $A_1, \dots, A_n$ are pairwise disjoint, each element of
	\(\bigcup_{i=1}^n A_i\) belongs to exactly one of the sets $A_i$.
	Thus no element is counted more than once when summing their cardinalities.
	Therefore,
	\[
	\left| \bigcup_{i=1}^n A_i \right|
	= |A_1| + |A_2| + \cdots + |A_n|.
	\]
\end{proof}

\begin{theorem}[Multiplication Principle]
	Let an experiment consist of $k$ stages, where stage $i$ admits $n_i$ possible
	outcomes for each outcome of the previous stages. Then the total number of
	possible outcomes of the experiment is
	\[
	n_1 n_2 \cdots n_k.
	\]
\end{theorem}


\begin{proof}
	Let $S_i$ denote the finite set of possible outcomes at stage $i$.
	An outcome of the experiment is determined by choosing one element from each
	$S_i$, so the sample space is the Cartesian product
	\[
	S = S_1 \times S_2 \times \cdots \times S_k.
	\]
	By the definition of Cartesian product,
	\[
	|S| = |S_1|\,|S_2|\cdots|S_k| = n_1 n_2 \cdots n_k.
	\]
\end{proof}


In general, we can depict the multiplication principle by a tree representing selection at each stage.
\begin{fullwidth}
	\begin{figure}[H]
		\centering
		\begin{tikzpicture}[
			grow=right,
			scale=0.8,
			transform shape,
			level distance=4.2cm,   % <-- THIS is what makes it span the page
			sibling distance=1.3cm,
			edge from parent/.style={draw, thick},
			every node/.style={font=\small},
			anchor=west
			]
			
			\node {Start}
			child { node {$a_1^{(1)}$}
				child { node {$a_2^{(1)}$}
					child { node {$a_3^{(1)}$}
						child { node {$\vdots$} }
						child { node {$a_k^{(1)}$} }
					}
					child { node {$\vdots$} }
					child { node {$a_3^{(n_3)}$} }
				}
				child { node {$\vdots$} }
				child { node {$a_2^{(n_2)}$} }
			}
			child { node {$\vdots$} }
			child { node {$a_1^{(n_1)}$} };
			
		\end{tikzpicture}
		
		\caption{A horizontal choice tree illustrating the Multiplication Principle.
			Each level represents a stage of choice, and each root-to-leaf path corresponds
			to a unique outcome.}
	\end{figure}
\end{fullwidth}