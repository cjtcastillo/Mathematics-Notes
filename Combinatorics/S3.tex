	Previously, for permutations we specifically cared about the ordering of the selections and distinguished them that way. Now, we look at the case for when we \textit{do not} consider the order. We derive the formula for \( \binom{n}{k} \) by separating the notions of
\emph{order} and \emph{choice}. Return to the original partial permutation formulation
\begin{equation*}
	P(n,k) = \frac{n!}{(n-k)!}
\end{equation*}

Each unordered \( k \)-element subset is counted once for every permutation of
its elements. Since a set of \( k \) elements has \( k! \) orderings, we divide by \( k! \) to
correct for overcounting/collapse all orderings to one count.
\begin{definition}
	For integers \( n,k \) with \( 0 \le k \le n \), the \textbf{binomial coefficient} is defined by the collection of all $k$-element subsets of a given set.
	\begin{equation}
			\binom{n}{k} := \frac{n!}{k!(n-k)!}.
	\end{equation}
\end{definition}
\begin{eg}\label{eg1}
	Consider the set $\set{A,B,C}$ and suppose we want to choose 2 letters, where order does not matter. The ordered selections of 2 distinct letters will look like:
	\begin{align*}
		(A,B)&,(B,A) \\
		(A,C)&,(C,A)\\
		(B,C)&,(C,B) 
	\end{align*}
	There are exactly 6 pairs. Notice that for each selection of 2 letters we have $2!$ permutations. Now using $2!$ to collapse ordered pairs to a subset 
	\begin{align*}
		(A,B)&,(B,A) \to \set{A,B} \\
		(A,C)&,(C,A) \to \set{A,C}\\
		(B,C)&,(C,B) \to \set{B,C}
	\end{align*}
	so only 3 distinct subsets remain.
	\begin{figure}[h!]
		\begin{center}
			\includegraphics[scale=.3]{BCE}
			\caption{(1) Initial selection (2) Results after permutations decided (3) After collapsing redundancies}
		\end{center}
	\end{figure}
\end{eg}

\begin{eg}
	Roll a six sided dice 10 total times. The total outcomes where two 1s are present is $\binom{10}{2}$. But now consider if we require that two 1s are present and consecutive in the order of the rolls. We treat the two 1s as a block. The total unique block slots out of 10 slots is 7. Then the outcomes with this condition is now $\binom{7}{2}$.
\end{eg}

\begin{theorem}[Identities]
	Binomial coefficients satisfy the following identities, 
	\begin{enumerate}[label=\roman*)]
		\item $\binom{n}{k} = \binom{n}{n-k}$
		\item $\binom{n}{k} = \binom{n-1}{k}+ \binom{n-1}{k-1}$
		\item $\binom{n}{0}+\binom{n}{1}+\cdots+\binom{n}{n-1}+\binom{n}{n} = 2^n$
	\end{enumerate}
\end{theorem}
\begin{proof}
	~
	\begin{enumerate}[label=\roman*)]
		\item
		Let $S$ be a set with $|S| = n$. Choosing a subset of $S$ with $k$ elements is equivalent
		to choosing which $n-k$ elements are excluded. Hence there is a bijection between
		$k$-element subsets and $(n-k)$-element subsets of $S$, so
		\[
		\binom{n}{k} = \binom{n}{n-k}.
		\]
		
		\item
		Let $S$ be a set with $|S| = n$ and fix an element $x \in S$.
		Every $k$-element subset of $S$ either contains $x$ or does not contain $x$.
		If it does not contain $x$, the $k$ elements are chosen from the remaining $n-1$
		elements, giving $\binom{n-1}{k}$ possibilities.
		If it does contain $x$, the remaining $k-1$ elements are chosen from the remaining
		$n-1$ elements, giving $\binom{n-1}{k-1}$ possibilities.
		Since these two cases are disjoint and exhaustive, we obtain
		\[
		\binom{n}{k} = \binom{n-1}{k} + \binom{n-1}{k-1}.
		\]
		
		\item
		Let $S$ be a set with $|S| = n$. The number of subsets of $S$ with exactly $k$ elements
		is $\binom{n}{k}$. Summing over all possible values of $k$ counts all subsets of $S$:
		\[
		\sum_{k=0}^n \binom{n}{k}.
		\]
		Alternatively, each element of $S$ may be either included or excluded from a subset,
		giving $2$ choices per element and hence $2^n$ subsets in total.
		Therefore,
		\[
		\binom{n}{0}+\binom{n}{1}+\cdots+\binom{n}{n} = 2^n.
		\]
	\end{enumerate}
\end{proof}