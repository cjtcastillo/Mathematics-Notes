	In the first section, we specifically examined pairwise disjoint sets to ensure proper behavior of counting. Now we expand on the addition principle to examine less ideal families of sets.
\begin{theorem}[Inclusion-Exclusion Principle]
	Let $A_1,..., A_n$ be finite sets. Then
	\[
	\left|\bigcup_{i=1}^n A_i\right|
	= \sum_{\emptyset \neq \mathcal{I} \subseteq \{1,2,\dots,n\}} 
	(-1)^{|\mathcal{I}|+1} \left|\bigcap_{i \in \mathcal{I}} A_i\right|.
	\]
\end{theorem}
\begin{remark}
	The expanded version of the formula above;
	\[
	\left|\bigcup_{i=1}^n A_i\right| 
	= \sum_{i=1}^{n} |A_i| 
	- \sum_{1 \le i < j \le n} |A_i \cap A_j| 
	+ \sum_{1 \le i < j < k \le n} |A_i \cap A_j \cap A_k| 
	- \cdots 
	+ (-1)^{n+1} |A_1 \cap A_2 \cap \cdots \cap A_n|.
	\]
\end{remark}
\begin{proof}
	We'll use mathematical induction. Assume that $A_i$ are non-pairwise disjoint sets. First, considering the trivial/base case for $n=2$, we know from the addition principle \sidenote{Also verify that this is true through directly plugging $n=2$ into equation.}
	\begin{equation*}
		|A_1 \cup A_2 | = |A_1|+|A_2| - |A \cap B|
	\end{equation*}
	
\end{proof}

