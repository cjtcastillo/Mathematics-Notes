\subsection{Ideals and Quotient Rings}
Previously, in group theory, we used normal subgroups to construct quotient groups. We now seek to develop analogous conditions for subrings to construct quotient rings. Recall that we may partition a ring through individual cosets that are \textit{equivalence classes} of the form
\begin{equation*}
	r+S = \set{r+s : s \in S}
\end{equation*}
If we define these to be the exact elements of a quotient ring $R/S$ we must have that 
\begin{align*}
	(r_1 + S)(r_2+S) &\in R/S \\
	(r_1+S)+(r_2+S) &\in R/S
\end{align*}
The second equation implies that $S$ must be closed under addition. The first equation implies that we must have
\begin{align*}
	(r_1+s)(r_2+s') &\in ab+S \\
	r_1r_2+r_1s'+sr_2+ss' & \in r_1r_2 + S
\end{align*}
Notice that $ss'$ is already in $S$ and $r_1r_2$ cancels out. This leaves us with
\begin{equation*}
	r_1s'+sr_2 \in S \imp r_1s' \in S \text{ and } sr_2 \in S
\end{equation*}
which describes the exact conditions needed for $S$ to be used to form a quotient ring.
\begin{definition}
	Let $R$ be a ring. A subset $I \subseteq R$ is called an \textbf{ideal} if:
	\begin{enumerate}[label=\roman*)]
		\item $I$ is closed under addition;
		\item for all $r \in R$ and $i \in I$, we have
		\[
		ri \in I \quad \text{and} \quad ir \in I.
		\]
	\end{enumerate}
	An ideal $I$ is called \textbf{proper} if $I \neq R$ (equivalently, $1 \notin I$). The ideal is considered to be \textbf{principal} if there exists $a \in R$ such that
	\begin{equation*}
		I = (a):= \set{ra: r \in R}
	\end{equation*}
\end{definition}

\newpage

\begin{definition}
	Let $R$ be a ring and let $I \unlhd R$ be an ideal. The \textbf{quotient ring}
	(or \textbf{factor ring}) $R/I$ is the set
	\[
	R/I := \{ r + I : r \in R \},
	\]
	with addition and multiplication defined by
	\[
	(r+I)+(s+I)=(r+s)+I,
	\qquad
	(r+I)(s+I)=rs+I.
	\]
\end{definition}
The formation of the quotient ring allows us to simplify structures of rings to analyze similarities and differences.

\begin{eg}
Let $R = \mathbb{R}[x]$.

The ideal generated by $x^2+1$ is
\[
(x^2+1) = \{(x^2+1)f(x) : f(x)\in \mathbb{R}[x]\}.
\]

The quotient ring
\[
\mathbb{R}[x]/(x^2+1)
\]
satisfies the relation $x^2 = -1$ and is isomorphic to $\mathbb{C}$.
\end{eg}


\begin{proposition}
	Let $\phi : R \to S$ be a ring homomorphism. Then $\ker(\phi)$ is an ideal of $R$.
\end{proposition}

\begin{proof}
	Let $a \in \ker(\phi)$ and $r \in R$. Then 
	\begin{equation*}
		\phi(ar) = \phi(a)\phi(r) = 0\phi(r) = 0, \quad \phi(ra) = \phi(r)\phi(a) = \phi(r)0 = 0
	\end{equation*}
	so $ar,ra \in \ker(\phi)$. Since $\ker(\phi)$ is a subring, it must be true that $\ker(\phi)$ is closed under addition. Thus, $\ker(\phi)$ is an ideal.
\end{proof}

\begin{corollary}
	Every ring homomorphism $\phi : K \to R$ from a field $K$ with $\phi \neq 0$ is injective.
\end{corollary}

\begin{proof}
	The kernel $\ker(\phi)$ is an ideal of $K$. Since fields have no proper ideals,
	we must have $\ker(\phi) = \{0\}$, hence $\phi$ is injective.
\end{proof}

\begin{proposition}
	A commutative ring $R$ is a field if and only if it has no proper ideals.
\end{proposition}

\begin{proof}
	$(\Rightarrow)$ If $R$ is a field, then every nonzero element is a unit.
	Thus any ideal containing a nonzero element must contain $1$, and hence must equal $R$.
	
	$(\Leftarrow)$ If $R$ has no proper ideals and $a \neq 0$, then $(a) = R$.
	Thus $1 \in (a)$, so there exists $b \in R$ with $ba = 1$, showing that $a$ is a unit.
	Hence $R$ is a field.
\end{proof}

Since every ideal is an additive subgroup of $R$, we may form the quotient group $R/I$.
Its elements are cosets of the form
\[
a + I = \{ a + i : i \in I \}.
\]
