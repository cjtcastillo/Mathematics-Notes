\subsection{Isomorphism Theorems}
\begin{theorem}[Canonical Projections]
	Let $R$ be a ring and $I \unlhd R$ be an ideal. Then define the canonical projection
	\begin{equation*}
		\pi:R \to R/I, \quad \pi(r) = r+I
	\end{equation*}
	Then 
	\begin{enumerate}
		\item $\pi$ is a homomorphism
		\item $\pi$ is surjective
		\item $\ker(\pi) = I$
	\end{enumerate}
\end{theorem}

\begin{proof}
	Let $r,s \in R$. 
	\begin{enumerate}
		\item Additivity:
		\begin{equation*}
			\pi(r+s) = (r+s) + I = (r+I)+(s+I) = \pi(r) + \pi(s)
		\end{equation*}
		\item Multiplicativity:
		\begin{equation*}
			\pi(rs) = (rs+I) = (r+I)(s+I) = \pi(r)\pi(s)
		\end{equation*}
		\item Identity: 
		\begin{equation*}
			\pi(1) = 1+I
		\end{equation*}
	\end{enumerate}
	Thus, $\pi$ is a homomorphism. Surjectivity is trivial in how we define the map. Now consider the kernel. By definition
	\begin{equation*}
		\ker(\phi) = \set{r \in R: \pi(r) = 0+I} = \set{r \in R: r \in I} = I 
	\end{equation*}
\end{proof}
\begin{theorem}[Mapping Property]
	Let $R,S$ be rings and let 
	\begin{equation*}
		\phi: R\to S
	\end{equation*}
	be a ring homomorphism. If $I \subseteq \ker(\phi)$, then there exists a unique ring homomorphism
	\begin{equation*}
		\overline{\phi}: R/I \to S
	\end{equation*}
	such that $\phi = \overline{\phi} \circ \pi $ where $\pi: R \to R/I$ is the canonical projection map. 
\end{theorem}

	Not every homomorphism sends elements of an ideal to $0$. The mapping property requires $I \subseteq \ker\phi$; only then does the quotient $R/I$ allow a well-defined factorization. Otherwise, cosets could map ambiguously.
\begin{figure}[h!]
	\begin{center}
		\includegraphics[scale=.3]{MAPPINGPROPERTY}
		\caption{Diagram of the mapping property in action}
	\end{center}
\end{figure}
\begin{proof}
	First, define the map 
	\begin{equation*}
		\overline{\phi} (r+I) = \phi(r)
	\end{equation*}
	Now suppose $r+I=r'+I$. Then 
	\begin{equation*}
		r-r' \in I \subseteq \ker(\phi)
	\end{equation*}
	so 
	\begin{equation*}
		\phi(r-r') = 0 \imp \phi(r)=\phi(r')
	\end{equation*}
	Hence, $\overline{\phi}$ is well defined. Now, let us confirm that the map is in fact a homomorphism. Let $r,s \in R$.
	\begin{enumerate}
		\item Additivity:
		\begin{equation*}
			\overline{\phi}((r+I)+(s+I)) = \overline{\phi}(r+s+I) = \phi(r+s) = \phi(r) + \phi(s) = \overline{\phi}(r+I)+\overline{\phi}(s+I)
		\end{equation*}
		\item Multiplicity: 
		\begin{equation*}
			\overline{\phi}((r+I)(s+I)) = \overline{\phi}(rs+I) = \phi(r)\phi(s) = \overline{\phi}(r+I)\overline{\phi}(s+I)
		\end{equation*}
		\item Identity:
		\begin{equation*}
			\overline{\phi}(1+I) = \phi(1) = 1
		\end{equation*}
	\end{enumerate}
	Thus, $\overline{\phi}$ is a ring homomorphism. So notice then that 
	\begin{equation*}
		(\overline{\phi}\circ \pi)(r) = \overline{\phi}(r+I) = \phi(r)
	\end{equation*}
	so $\phi = \overline{\phi}\circ \pi$. Now, suppose $\psi:R/I \to S$ is any ring homomorphism with $\phi = \psi \circ \pi$. Then for any $r+I \in R/I$,
	\begin{equation*}
		\psi(r+I) = \psi(\pi(r)) = \phi(r) = \overline{\phi}(r+I)
	\end{equation*}
	Thus, $\psi = \overline{\phi}$, proving uniqueness.
\end{proof}
\begin{corollary}[First Isomorphism Theorem]
	Let $\phi: R\to S$ be a ring homomorphism. Then 
	\begin{equation*}
		R/\ker(\phi) \cong \text{im}\phi
	\end{equation*}
\end{corollary}
	The use of the kernel as the ideal to quotient by is key for inducing injectivity, as all elements that map to 0 are collapsed to a singular representation.
\begin{theorem}[Correspondence Theorem]
	Let $R$ be a ring and let $I \unlhd R$ be an ideal. Then there is a one to one correspondence between
	\begin{enumerate}
		\item Ideals of $R$ containing $I$
		\item Ideals of the quotient ring $R/I$
	\end{enumerate}
	The correspondence is given by 
	\begin{equation*}
		J \mt J/I = \set{j+I: j \in J}, \quad \overline{J} \mt \pi^{-1}(\overline{J}) = \set{r \in R: r+I \in J}
	\end{equation*}
\end{theorem}
\begin{proof}
	Let $R$ be a ring and $I \unlhd R$ be an ideal. Then let $J$ be an ideal of $R$ such that $I \subseteq J$. Define a map
	\begin{equation*}
		\psi(r) = r+I
	\end{equation*}
	Return to $J$. Under the map $\psi$, we have that $J$ becomes
	\begin{equation*}
		J/I = \set{j+I: j \in J}
	\end{equation*}
	We'll now show that this resulting set is an ideal of $R/I$. Let $r+I \in R/I$ and $j+I \in J$. Then we have that
	\begin{equation*}
		(r+I)(j+I) = (rj+I)
	\end{equation*}
	Because $J$ was an ideal, we have that $rj \in J$. Thus, it must be true that $rj+I \in J/I$. Therfore, $J/I$ is absorbs via multiplication. Next, let $j_1+I,j_2+I \in J/I$. Then
	\begin{equation*}
		j_1+I+j_2+I = (j_1+j_2)+I
	\end{equation*}
	Since $J$ is an ideal, $j_1+j_2 \in J$. Thus, $J/I$ is closed under addition. Therfore, $J/I$ is an ideal.
	Thus, $J/I$ is an ideal. Conversely, if $K \unlhd R/I$, then the preimage
	\[
	\pi^{-1}(K) = \{ r \in R : r+I \in K \}
	\]
	is an ideal of $R$ containing $I$. Moreover, these assignments are
	inverse to each other:
	\[
	\pi^{-1}(J/I) = J \quad \text{and} \quad \pi(\pi^{-1}(K)) = K.
	\]
	Hence there is a bijection between ideals of $R$ containing $I$ and
	ideals of $R/I$, given by
	\[
	J \longleftrightarrow J/I.
	\]
\end{proof}